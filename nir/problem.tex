\chapter{Анализ предметной области}

\section{Задача распознавания образов}

Распознавание изображений - научное направление, связанное с разработкой принципов и построением систем, предназначенных для определения принадлежности данного объекта к одному из заранее выделенных классов объектов.

Распознавание изображений — это метод компьютерного зрения для идентификации объектов на изображениях или видео. Распознавание изображений является основным результатом алгоритмов глубокого и машинного обучения. При просмотре фотографий или видео, человек может легко распознать людей, предметы, сцены и визуальные детали. Цель состоит в обучении компьютера делать то, что естественно для людей: достичь уровня понимания того, что содержит изображение.

Алгоритм распознавания изображений (также известный как классификатор изображений) принимает изображение (или фрагмент изображения) в качестве входных данных и выводит то, что содержит изображение. Другими словами, вывод — это метка класса (например, «кошка», «собака», «таблица» и т.д.).

Существует несколько специализированных задач, основанных на распознавании, например:
\begin{itemize}
	\item Поиск изображений по содержанию: нахождение всех изображений в большом наборе изображений, которые имеют определённое содержание. Содержание может быть определено различными путями, например в терминах схожести с конкретным изображением (найдите мне все изображения похожие на данное изображение), или в терминах высокоуровневых критериев поиска, вводимых как текстовые данные (найдите мне все изображения, на которых изображено много домов, которые сделаны зимой и на которых нет машин).
	\item Оценка положения: определение положения или ориентации определённого объекта относительно камеры. Примером применения этой техники может быть содействие руке робота в извлечении объектов с ленты конвейера на линии сборки.
	\item Оптическое распознавание знаков: распознавание символов на изображениях печатного или рукописного текста, обычно для перевода в текстовый формат, наиболее удобный для редактирования или индексации (например, ASCII).
\end{itemize}
\section{Возможные сложности}

Классическая задача в распознавании изображений - определение содержат ли видеоданные некоторый характерный объект, особенность или активность. Эта задача может быть достоверно и легко решена человеком, но до сих пор не решена удовлетворительно в компьютерном зрении в общем случае: случайные объекты в случайных ситуациях.

Кроме того есть некоторые проблемы с распознаванием изображений:
\begin{itemize}
	\item Изменение точки зрения.
	\item Изменение освещения.
	\item Деформация.
	\item Изображения частично скрыто.
	\item Изображения совпадает с фоном.
\end{itemize}

