\chapter{Предобработка исходных изображений}

\begin{itemize}
\item Часто входное изображение предварительно обрабатывается для нормализации эффектов контрастности и яркости. Очень распространенный этап предварительной обработки — вычесть среднее значение интенсивности изображения и разделить его на стандартное отклонение. Иногда гамма-коррекция дает немного лучшие результаты. При работе с цветными изображениями преобразование цветового пространства (например, цветовое пространство RGB в LAB) может помочь получить лучшие результаты.

В рамках предварительной обработки входное изображение или фрагмент изображения также обрезаются и изменяются до фиксированного размера. Это важно, потому что следующий шаг, извлечение признаков, выполняется на изображении фиксированного размера.

\item Входное изображение содержит слишком много дополнительной информации, которая не нужна для классификации. Следовательно, первым шагом в классификации изображений является упрощение изображения путем извлечения важной информации, содержащейся в изображении, и исключения остальной части.  Этот шаг называется извлечением признаков. В традиционных подходах к компьютерному зрению разработка этих функций имеет решающее значение для производительности алгоритма. 
\end{itemize}

Полученное изображение после предварительной обработки будет использоваться в алгоритмах/методах классификации изображений.
